\documentclass[a4paper]{article}
\usepackage[T1]{fontenc}
\usepackage{array}
\usepackage{booktabs}
\usepackage{siunitx}
\usepackage{tabularx}
%\usepackage[paperheight=8cm,paperwidth=8cm]{geometry}
\usepackage{longtable}
\usepackage{threeparttable}
\usepackage{ragged2e}
\usepackage[russian]{babel}
\usepackage{caption}

\captionsetup{labelfont=bf}

\begin{document}

\begin{tabular}{lll}
Animal & Food & Size \\
dog
& meat & medium \\
horse & hay & large \\
frog & flies & small \\
\end{tabular}

\begin{tabular}{cl}
Animal & Description \\
dog    & The dog is a member of the genus Canis, which forms part of the wolf-like canids, and is the most widely abundant terrestrial carnivore. \\
cat    & The cat is a domestic species of small carnivorous mammal. It is the only domesticated species in the family Felidae and is often referred to as the domestic cat to distinguish it from the wild members of the family. \\
\end{tabular}

\begin{tabular}{cp{9cm}}
Animal & Description \\
dog    & The dog is a member of the genus Canis, which forms part of the wolf-like canids, and is the most widely abundant terrestrial carnivore. \\
cat    & The cat is a domestic species of small carnivorous mammal. It is the only domesticated species in the family Felidae and is often referred to as the domestic cat to distinguish it from the wild members of the family. \\
\end{tabular}

\begin{tabular}{*{3}{l}}
  Animal & Food  & Size   \\
  dog    & meat  & medium \\
  horse  & hay   & large  \\
  frog   & flies & small  \\
\end{tabular}

\begin{tabular}{lll}
  \toprule
  Animal & Food & Size   \\
  \midrule
  dog   & meat  & medium \\
  horse & hay   & large  \\
  frog  & flies & small  \\
  \bottomrule
\end{tabular}

\begin{tabular}{lll}
  \toprule
  Animal & Food & Size \\
  \midrule
  dog & meat & medium \\
  \cmidrule{1-2}
  horse & hay & large \\
  \cmidrule{1-1}
  \cmidrule{3-3}
  frog & flies & small \\
  \bottomrule
\end{tabular}

\begin{tabular}{lll}
  \toprule
  Animal & Food & Size \\
  \midrule
  dog & meat & medium \\
  \cmidrule{1-2}
  horse & hay & large \\
  \cmidrule(r){1-1}
  \cmidrule(rl){2-2}
  \cmidrule(l){3-3}
  frog & flies & small \\
  \bottomrule
\end{tabular}

\begin{tabular}{cp{9cm}}
  \toprule
  Animal & Description \\
  \midrule
  dog    & The dog is a member of the genus Canis, which forms part of the wolf-like canids, and is the most widely abundant terrestrial carnivore. \\
  \addlinespace
  cat    & The cat is a domestic species of small carnivorous mammal. It is the only domesticated species in the family Felidae and is often referred to as the domestic cat to distinguish it from the wild members of the family. \\
  \bottomrule
\end{tabular}

\begin{tabular}{lll}
  \toprule
  Animal & Food & Size \\
  \midrule
  dog & meat & medium \\
  horse & hay & large \\
  frog & flies & small \\
  fuath & \multicolumn{2}{c}{unknown} \\
  \bottomrule
\end{tabular}

\begin{tabular}{lll}
  \toprule
  \multicolumn{1}{c}{Animal} & \multicolumn{1}{c}{Food} & \multicolumn{1}{c}{Size} \\
  \midrule
  dog & meat & medium \\
  horse & hay & large \\
  frog & flies & small \\
  fuath & \multicolumn{2}{c}{unknown} \\
  \bottomrule
\end{tabular}

\begin{tabular}{lll}
  \toprule
  Group & Animal & Size \\
  \midrule
  herbivore & horse & large \\
  & deer & medium \\
  & rabbit & small \\
  \addlinespace
  carnivore & dog & medium \\
  & cat & small \\
  & lion & large \\
  \addlinespace
  omnivore & crow & small \\
  & bear & large \\
  & pig & medium \\
  \bottomrule
\end{tabular}

\begin{tabular}{>{\itshape}l<{:} *{2}{l}}
  \toprule
  Animal & Food & Size \\
  \midrule
  dog & meat & medium \\
  horse & hay & large \\
  frog & flies & small \\
  \bottomrule
\end{tabular}

\begin{tabular}{>{\itshape}l<{:} *{2}{l}}
  \toprule
  \multicolumn{1}{l}{Animal} & Food & Size \\
  \midrule
  dog & meat & medium \\
  horse & hay & large \\
  frog & flies & small \\
  \bottomrule
\end{tabular}

\begin{tabular}{lll}
 Animal & Food & Size \\
 dog & meat & medium \\
 horse & hay & large \\
 frog & flies & small \\
\end{tabular}

\begin{tabular}{l@{ : }l@{\hspace{2cm}}l}
  Animal & Food & Size \\
  dog & meat & medium \\
  horse & hay & large \\
  frog & flies & small \\
\end{tabular}

\begin{tabular}{l!{:}ll}
  Animal & Food & Size \\
  dog & meat & medium \\
  horse & hay & large \\
  frog & flies & small \\
\end{tabular}

\begin{tabular}{l|ll}
 Animal & Food & Size \\[2pt]
 dog & meat & medium \\
 horse & hay & large \\
 frog & flies & small \\
\end{tabular}

\begin{tabular}{@{} lll@{}} \toprule[2pt]
  Animal & Food & Size \\ \midrule[1pt]
  dog & meat & medium \\
  \cmidrule[0.5pt](r{1pt}l{1cm}){1-2}
  horse & hay & large \\
  frog & flies & small \\ \bottomrule[2pt]
\end{tabular}

\begin{tabular}{SS}
  \toprule
  {Values} & {More Values} \\
  \midrule
  1    & 2.3456  \\
  1.2  & 34.2345 \\
  -2.3 & 90.473  \\
  40   & 5642.5  \\
  5.3  & 1.2e3   \\
  0.2  & 1e4     \\
  \bottomrule
\end{tabular}

\begin{center}
\begin{tabular}{cc}
\hline
A & B\\
C & D\\
\hline
\end{tabular}
\end{center}

\begin{center}
\begin{tabular*}{.5\textwidth}{@{\extracolsep{\fill}}cc@{}}
\hline
A & B\\
C & D\\
\hline
\end{tabular*}
\end{center}

\begin{center}
\begin{tabular*}{\textwidth}{@{\extracolsep{\fill}}cc@{}}
\hline
A & B\\
C & D\\
\hline
\end{tabular*}
\end{center}

\begin{center}
\begin{tabular}{lp{2cm}}
\hline
A & B B B B B B B B B B B B B B B B B B B B B B B B\\
C & D D D D D D D\\
\hline
\end{tabular}
\end{center}

\begin{center}
\begin{tabularx}{.5\textwidth}{lX}
\hline
A & B B B B B B B B B B B B B B B B B B B B B B B B\\
C & D D D D D D D\\
\hline
\end{tabularx}
\end{center}

\begin{center}
\begin{tabularx}{\textwidth}{lX}
\hline
A & B B B B B B B B B B B B B B B B B B B B B B B B\\
C & D D D D D D D\\
\hline
\end{tabularx}
\end{center}

%\begin{longtable*}{cc}
%\multicolumn{2}{c}{A Long Table}\\
%Left Side & Right Side\\
%\hline
%\endhead
%\hline
%\endfoot
%aa & bb\\
%Entry & b\\
%a & b\\
%a & b\\
%a & b\\
%a & b\\
%a & bbb\\
%a & b\\
% & b\\
%a & b\\
%a & b\\
%a & b\\
%a & b\\
%a & b b b b b b\\
%a & b b b b b\\
%a & b b\\
%A Wider Entry & b\\
%\end{longtable*}


\begin{table}
\begin{threeparttable}
\caption{An Example}
\begin{tabular}{ll}
An entry & 42\tnote{1}\\
Another entry & 24\tnote{2}\\
\end{tabular}
\begin{tablenotes}
\item [1] the first note.
\item [2] the second note.
\end{tablenotes}
\end{threeparttable}
\end{table}

\begin{table}
\begin{tabular}[t]{lp{3cm}}
One & A long text set in a narrow paragraph, with some more example text.\\
Two & A different long text set in a narrow paragraph, with some more hard to hyphenate words.
\end{tabular}%
\begin{tabular}[t]{l>{\raggedright\arraybackslash}p{3cm}}
One & A long text set in a narrow paragraph, with some more example text.\\
Two & A different long text set in a narrow paragraph, with some more hard to hyphenate words.
\end{tabular}%
\begin{tabular}[t]{l>{\RaggedRight}p{3cm}}
One & A long text set in a narrow paragraph, with some more example text.\\
Two & A different long text set in a narrow paragraph, with some more hard to hyphenate words.
\end{tabular}

\footnotesize
\begin{tabular}[t]{lp{3cm}}
One & A long text set in a narrow paragraph, with some more example text.\\
Two & A different long text set in a narrow paragraph, with some more hard to hyphenate words.
\end{tabular}
\end{table}

\begin{tabular}{lcc}
  \toprule
  Test & \begin{tabular}{@{}c@{}}A\\a\end{tabular} & \begin{tabular}{@{}c@{}}B\\b\end{tabular} \\
  \midrule
  Content & is & here \\
  Content & is & here \\
  Content & is & here \\
  \bottomrule
\end{tabular}

\begin{tabular}{lcc}
  \toprule
  Test & \begin{tabular}[b]{@{}c@{}}A\\a\end{tabular} & \begin{tabular}[t]{@{}c@{}}B\\b\end{tabular} \\
  \midrule
  Content & is & here \\
  Content & is & here \\
  Content & is & here \\
  \bottomrule
\end{tabular}

\begin{center}
\begin{tabular}{cc}
\hline
Square& $x^2$\\
\hline
Cube& $x^3$\\
\hline
\end{tabular}
\end{center}
 \begin{center}
\setlength\extrarowheight{2pt}
\begin{tabular}{cc}
\hline
Square& $x^2$\\
\hline
Cube& $x^3$\\
\hline
\end{tabular}
\end{center}

\section*{Упражнения по таблицам}

\subsection*{1. Простой пример таблицы}
\begin{center}
\begin{tabular}{lll}
\toprule
Животное & Еда & Размер \\
\midrule
Собака & мясо & средний \\
Лошадь & сено & большой \\
Лягушка & мухи & маленький \\
\bottomrule
\end{tabular}
\end{center}

\subsection*{2. Выравнивания столбцов: \texttt{l}, \texttt{c}, \texttt{r}}
\paragraph{(a) \texttt{lll} — всё по левому краю}
\begin{center}
\begin{tabular}{lll}
\toprule
A & B & C \\
\midrule
alpha & beta & gamma \\
\bottomrule
\end{tabular}
\end{center}

\paragraph{(b) \texttt{ccc} — центр}
\begin{center}
\begin{tabular}{ccc}
\toprule
A & B & C \\
\midrule
alpha & beta & gamma \\
\bottomrule
\end{tabular}
\end{center}

\paragraph{(c) \texttt{rrr} — по правому краю}
\begin{center}
\begin{tabular}{rrr}
\toprule
A & B & C \\
\midrule
alpha & beta & gamma \\
\bottomrule
\end{tabular}
\end{center}

\subsection*{3. Меньше элементов, чем столбцов}
\paragraph{Корректный вариант 1: добавить пустую ячейку}
\begin{center}
\begin{tabular}{lll}
\toprule
A & B & C \\
\midrule
1 & 2 & {} \\
\bottomrule
\end{tabular}
\end{center}

\paragraph{Корректный вариант 2: объединить оставшиеся столбцы}
\begin{center}
\begin{tabular}{lll}
\toprule
A & B & C \\
\midrule
1 & \multicolumn{2}{l}{третий столбец отсутствует} \\
\bottomrule
\end{tabular}
\end{center}

\subsection*{4. Больше элементов, чем столбцов}

\paragraph{Корректный вариант: объединить две последние ячейки}
\begin{center}
\begin{tabular}{lll}
\toprule
A & B & C \\
\midrule
1 & 2 & \multicolumn{1}{l}{3 и 4 как текст} \\
\bottomrule
\end{tabular}
\end{center}

\paragraph{Альтернатива: добавить столбец и заполнить}
\begin{center}
\begin{tabular}{llll}
\toprule
A & B & C & D \\
\midrule
1 & 2 & 3 & 4 \\
\bottomrule
\end{tabular}
\end{center}

\subsection*{5. Эксперименты с \texttt{\textbackslash multicolumn}}

\paragraph{Объединение нескольких столбцов в ячейке}
\begin{center}
\begin{tabular}{llll}
\toprule
\multicolumn{2}{c}{Левая часть} & \multicolumn{2}{c}{Правая часть} \\
\midrule
L1 & L2 & R1 & R2 \\
\multicolumn{2}{l}{Итого слева} & \multicolumn{2}{r}{Итого справа} \\
\bottomrule
\end{tabular}
\end{center}

\end{document}