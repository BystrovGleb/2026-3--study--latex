\documentclass{article}

% Для многоязычности
\usepackage{polyglossia}
\setdefaultlanguage[indentfirst=true,spelling=modern]{russian}
\setotherlanguage{english}
% Юникодные математические символы
\usepackage{unicode-math}

% Подключаем шрифт. Шрифт есть в дистрибутиве TeXLive
\setmainfont[Ligatures={Common,TeX},Scale=0.94]{IBM Plex Serif}
\setromanfont[Ligatures={Common,TeX},Scale=0.94]{IBM Plex Serif}
\setsansfont[Ligatures={Common,TeX},Scale=MatchLowercase,Scale=0.94]{IBM Plex Sans}
\setmonofont[Scale=MatchLowercase,Scale=0.94,FakeStretch=0.9]{IBM Plex Mono}

% Математический шрифт
\setmathfont{STIX Two Math}

% Пакет для подключения картинок
\usepackage{graphicx}
\usepackage{float}
% Пакет для ссылок (hyper references)
\usepackage{hyperref}


\author{Быстров Глеб Андреевич}
\title{Отчет №2}

% Для компиляции выполнить pdflatex main.tex

\begin{document}
\maketitle
\pagebreak

\section*{Цель работы}
Ознакомиться со структурой документа в \LaTeX{}, научиться создавать простейший файл, добавлять абзацы,
использовать комментарии (\%), жёсткие пробелы (\verb|~|), а также собрать PDF с помощью \texttt{xelatex}.

\section*{Задание}
Создать документ \texttt{exercise\_gabystrov.tex} со структурой преамбулы и тела документа, 
набрать несколько абзацев, продемонстрировать жёсткие пробелы и комментарии; скомпилировать документ и получить PDF.

\section*{Ход работы}

\subsection*{Шаг 1. Подготовка исходного файла}
Был создан исходный файл \texttt{exercise\_gabystrov.tex} и набран текст документа 
с несколькими абзацами, комментариями и примерами жёстких пробелов (рис.~\ref{img1}).

\begin{figure}[H]
    \centering
    \includegraphics[width=0.95\linewidth]{1.png}
    \caption{Исходный текст документа в редакторе}
    \label{img1}
\end{figure}

\subsection*{Шаг 2. Компиляция документа}
Документ скомпилирован в терминале командой \texttt{xelatex exercise\_gabystrov.tex}. 
На рисунке показан ход компиляции (рис.~\ref{img2}).

\begin{figure}[H]
    \centering
    \includegraphics[width=0.95\linewidth]{2.png}
    \caption{Компиляция документа с помощью \texttt{xelatex} в PowerShell}
    \label{img2}
\end{figure}

\subsection*{Шаг 3. Просмотр результата}
Получен итоговый PDF-документ, в котором видны абзацы, переносы строк и влияние жёстких пробелов
на разрыв строк (рис.~\ref{img3}).

\begin{figure}[H]
    \centering
    \includegraphics[width=0.8\linewidth]{3.png}
    \caption{Результат компиляции в формате PDF}
    \label{img3}
\end{figure}

\section*{Выводы}
Создан и собран минимальный документ в \LaTeX{}. 
Отработаны приёмы: организация структуры (преамбула/тело), добавление абзацев, комментариев и жёстких пробелов,
а также компиляция исходного файла в PDF с помощью \texttt{xelatex}.

\end{document}