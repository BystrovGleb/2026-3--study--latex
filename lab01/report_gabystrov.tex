\documentclass{article}

% Для многоязычности
\usepackage{polyglossia}
\setdefaultlanguage[indentfirst=true,spelling=modern]{russian}
\setotherlanguage{english}
% Юникодные математические символы
\usepackage{unicode-math}

% Подключаем шрифт. Шрифт есть в дистрибутиве TeXLive
\setmainfont[Ligatures={Common,TeX},Scale=0.94]{IBM Plex Serif}
\setromanfont[Ligatures={Common,TeX},Scale=0.94]{IBM Plex Serif}
\setsansfont[Ligatures={Common,TeX},Scale=MatchLowercase,Scale=0.94]{IBM Plex Sans}
\setmonofont[Scale=MatchLowercase,Scale=0.94,FakeStretch=0.9]{IBM Plex Mono}

% Математический шрифт
\setmathfont{STIX Two Math}

% Пакет для подключения картинок
\usepackage{graphicx}
\usepackage{float}
% Пакет для ссылок (hyper references)
\usepackage{hyperref}


\author{Быстров Глеб Андреевич}
\title{Отчет}

% Для компиляции выполнить pdflatex main.tex

\begin{document}
\maketitle
\pagebreak
\section*{Введение}
В данной лабораторной работе я устанавливал дистрибутив \LaTeX{} --- TeX Live на операционную систему Windows. 
Целью работы было ознакомиться с процессом установки и конфигурации \LaTeX{} на персональном компьютере для дальнейшего использования 
при подготовке научных и учебных отчётов. 

\section*{Процесс выполнения лабораторной работы}
Для выполнения работы я следовал официальной инструкции по установке TeX Live. Процесс установки включал несколько шагов, 
подтверждённых скриншотами. 

\subsection*{Шаг 1. Скачивание установочного файла}
Сначала был загружен установочный файл TeX Live для Windows (рис.~\ref{img1}).

\begin{figure}[H]
    \centering
    \includegraphics[width=0.9\linewidth]{1.png}
    \caption{Загрузка установочного файла TeX Live для Windows}
    \label{img1}
\end{figure}

\subsection*{Шаг 2. Запуск установщика}
После запуска установочного файла открылось окно выбора типа установки (рис.~\ref{img2}).

\begin{figure}[H]
    \centering
    \includegraphics[width=0.7\linewidth]{2.png}
    \caption{Окно установщика TeX Live (выбор режима установки)}
    \label{img2}
\end{figure}

\subsection*{Шаг 3. Настройка параметров установки}
Были выбраны каталог установки и параметры (по умолчанию) для установки редактора TeXworks (рис.~\ref{img3}).

\begin{figure}[H]
    \centering
    \includegraphics[width=0.9\linewidth]{3.png}
    \caption{Окно выбора параметров установки TeX Live}
    \label{img3}
\end{figure}

\subsection*{Шаг 4. Установка пакетов}
Процесс установки включал поочередную загрузку и установку множества пакетов \LaTeX{}, что заняло продолжительное время (рис.~\ref{img4}).

\begin{figure}[H]
    \centering
    \includegraphics[width=0.9\linewidth]{4.png}
    \caption{Установка пакетов TeX Live}
    \label{img4}
\end{figure}

\subsection*{Шаг 5. Завершение установки}
На завершающем этапе были выполнены действия по настройке системы, генерации кэшей и завершению установки (рис.~\ref{img5}).

\begin{figure}[H]
    \centering
    \includegraphics[width=0.9\linewidth]{5.png}
    \caption{Финальный этап установки TeX Live}
    \label{img5}
\end{figure}

\section*{Выводы}
В ходе лабораторной работы была успешно выполнена установка дистрибутива \LaTeX{} --- TeX Live 2025 на Windows. 
Я ознакомился с процессом скачивания установщика, выбора параметров, установки пакетов и завершения конфигурации системы. 
Теперь мой компьютер готов к созданию отчётов и других документов с использованием \LaTeX{}. 

\end{document}